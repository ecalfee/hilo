\documentclass[12pt]{report}
\usepackage{amsmath}
\begin{document}

\section{Methods}
These are the methods for the HiLo maize-mexicana admixture project.

\subsection{DNA extraction, library preparation, pooling/barcoding etc.}
\subsection{Sequencing}
Fourteen pairs of sympatric maize (landrace) and mexicana populations, twelve individuals per population, were genotyped using low-coverage whole-genome-sequencing (mean **x - **x). These populations span a range of altitudes from ***m to ***m in Mexico (see map, Fig. **). In addition, ten mexicana individuals from each of three allopatric populations were sequenced as non-admixed reference panel for mexicana ancestry and sequenced twice for additional depth (**x - **x mean individual coverage). Of the individuals sequenced in this study, ** were excluded due to low library amplification (< **x coverage), leaving a total of ** sympatric maize paired with ** sympatric mexicana and *** reference allopatric mexicana. For a maize reference panel, we use the publicly available 282 maize diversity panel which includes lines from *** and **. From these lines, we excluded ** and **. Mexicana is known from other sequencing efforts to be introgressed into maize from the highlands of Mexico and Guatemala, and to a lesser extent, the southwestern USA (Wang 2017). \par
\subsection{Alignment and Filtering}
We mapped reads using bwa to the AGPv4 maize genome ([bwa command; \& cite ABPv4]). For the reads that mapped (n= **, **\%), we applied additional filtering to remove reads with low mapping quality (n = **) [samtools 30?], remove likely PCR duplicates (n = **) [samtools not picard blah blah], and filter by adjusted base quality scores (BAQ > 20 in samtools)[NO local re-alignment around indels GATK (?)]. We then removed indels and sites that are invariant across our sequenced individuals and the maize reference panel. Of the total ** SNPs in the final variant set, ** vary within the maize reference panel only.
\subsection{Alignment and Filtering methods from Wang 2017}
Wang, Beissinger, Lorant, Ross-Ibarra, Ross-Ibarra, Hufford 2017. The interplay of demography and selection during maize domestication and expansion.\par
Sequenced 31 maize landraces from across the Americas and 4 parviglumis from Balsas River Valley, Mexico.\par
"Library preparation and Illumina HiSeq 2000 sequencing (100-bp paired-end) were conducted by BGI (Shenzhen, China) following their established protocols. the Burrows-Wheeler Aligner (BWA) v.0.7.5.a [73] was used to map reads to the maize B73 reference genome v3 (GenBank BioProject PRJNA72137) [74] with default settings. The duplicate molecules in the realigned bam files were removed with MarkDuplicates in Picardtools v.1.106 (http://broadinstitute.github.io/picard), and indels were realigned with the Genome Analysis Toolkit (GATK) v.3.3-0 [75]. Sites with mapping quality less than 30 and base quality less than 20 were removed, and only uniquely mapped reads were included in downstream analyses."\par
They have high 24-53X coverage per individual, which may make some SNP calling methods better for their study that don't apply to the current study. We will not be able to phase haplotypes and I am not sure that HaplotypeCaller is an option with low coverage data because it's like a local assembly (also newest release says you don't need to do GATK indel re-alignment if you do HaplotypeCaller method).\par
"SNPs were called via HaplotypeCaller and filtered via VariantFiltration in GATK [75] across all samples. SNPs with the following metrics were excluded from the analysis: QD <2.0; FS >60.0; MQ <40.0; MQRankSum <−12.5; ReadPosRankSum <−8.0. Vcftools v.0.1.12 [76] was used to further filter SNPs to include only bi-allelic sites. Following these data filtering steps, our data set consisted of 49 million SNPs. SNPs were phased using BEAGLE v.4.0 [77] with SNP data from the maize HapMap2 panel [78] used as a reference. Only sites with depth between half and twice of the mean depth were included in analyses. In addition, the software SNPable (http://lh3lh3.users.sourceforge.net/snpable.shtml) was used to mask genomic regions in which reads were not uniquely mapped. The mappability mask file for MSMC was generated by stepping in 1-bp increments across the maize genome to generate 100-bp single-end reads, which were then mapped back to the maize B73 reference genome [74]. Sites with the majority of overlapping 100-mers mapped uniquely without mismatch were determined to be “SNPable” sites and used for the MSMC analyses."\par

\subsection{Ancestry inference}
To infer global ancestry genomewide and local ancestry along the genome, we used NGSadmix and ancestry\_hmm respectively, chosen because they account for the genotype uncertainty from low-coverage sequencing. We used a subset of the total SNPs to infer ancestry because these methods assume independent information about ancestry from each marker and therefore markers in linkage disequilibrium within ancestry confound inference.\par
For global ancestry inference, randomly subsample SNPs to a set of 25k with high coverage across individuals and maf > .01 and input beagle genotype likelihoods for these spots from all individuals sequenced for this study to NGSadmix. We can repeat this procedure with 10 different SNP subsets to confirm similar results. This process doesn't include the maize reference or indicate which individuals are the unadmixed teosinte and I can use it to get a second estimate of allele frequencies in maize and do PCR with that against the reference population. We expect a gradient of admixture from low mexicana ancestry in the lowlands and high mexicana ancestry in the highlands and little to no maize ancestry in the mexicana allopatric panel.\par

For local ancestry inference, we need more SNPs, but still with low LD in the ancestral population. However, rather than a random subset, we want to keep SNPs with high ability to distinguish between mexicana and maize ancestry. To reduce the total number of SNPs, we first filtered out SNPs with low ancestry informativeness (difference in frequency between mexicana and maize < 0.2). From the remaining ***number** markers, we reduced our SNPs by using a greedy algorithm in plink that removes SNPs in high LD in a sliding window of 50 variants ([plink code snipet here]), resulting in a final variant set of *n=** SNPs. Then we input allele counts for reference populations and read counts supporting alternate and reference alleles for each admixed individual and run ancestry\_hmm for one admixed population at a time with whole-genome data. We estimate time since admixture to be at least 1 generations ago and at most 9000 generations in the past at the time of maize domestication assuming 1 gen/year (Matsuoka 2002) [**alt I could use 200 generations as the upper limit because 170ish is the upper of what Hufford 2013 found**], using the mean global ancestry proportion from NGSadmix as the proportion of migrants entering the population. Recombination distance between markers was calculated using the 0.2cM scale recombination map produced from 25 families in the maize Nested Association Mapping (NAM) population (Ogut 2015).\par

Our allopatric mexicana population was not sequenced to a high enough depth to accurately call heterogygous vs. homozygous genotypes. Therefore, to estimate an unbiased allele frequency, we randomly sampled one read at each SNP from each low-coverage allopatric mexicana individual and used these as our reference allele counts, essentially creating **number** unbiased pseudo-haplotypes. The maize reference allele counts are more certain and taken from the 282 panel with weighting by phylogenetic dissimilarity (?) to reduce the impact of biased sampling.\par

We assess confidence in our ancestry calls by calculating and comparing divergence to our reference panels in regions with high confidence mexicana or maize ancestry. This is a partially independent assessment from the local ancestry calling because we use all SNPs, **\% more than were used in ancestry\_hmm due to filtering for high LD [see supplemental fig. X for posterior prob. maize <-> mex on x axis and mean divergence on y axis, with separate lines for maize and mexicana .. I could also do this for divergence to specific mexicana groups]. We could alternatively use all the SNPs except the ones used for ancestry inference. \par

\subsection{zTz statistic}
The zTz statistic is the squared Mahalanobis distance between the observed population ancestry frequencies at a particular locus and the set of ancestry distributions across the whole genome, taking into account covariances caused by shared drift between populations. Our underlying assumption is that most loci are neutral, and therefore the genome-wide distribution represents neutral demographic history. zTz at a locus $l$ is defined 
$zTz=(anc - \alpha)^{T}K^{-1}(anc - \alpha)$,
where $anc$ is the vector of population ancestry frequencies, $\alpha$ is the genome-wide mean population ancestry frequency, and $K$ is the ancestry variance-covariance matrix,
$K = blah blah$.

If ancestry frequencies are independent across populations, all of the off-diagonal elements of K are zero, and the zTz statistic simplifies to the standardized Euclidean distance between observed population ancestries and their genome-wide means. When admixed populations are related to one another through a simple bifurcating tree, e.g. a historical admixed population that has split into two observed present-day populations, the off-diagonal elements of K are proportional to the branch lengths of that population tree. In the more general case, the off-diagonal elements of K capture shared drift between populations as a result of either gene flow or population splits after admixture began. Importantly, by ignoring kinship below the level of ancestry, K only captures shared drift post-admixture, and not kinship that pre-dates admixture, which would otherwise confound tests for selection on ancestry. When the genome-wide ancestry distribution follows a neutral multivariate normal model of drift with $j$ observed populations, zTz is chi-squared distributed with $j$ degrees of freedom. 

We use zTz to identify outlier loci whose observed ancestry frequencies across populations are improbable under neutral demographic history and drift alone, and these are our top candidates for selected loci. 

\subsection{Similarity to previous models and statistics}
We start with a model of admixture..Long 1991.
We merge these sophisticated techniques for accounting for shared population relationships with a separate body of work detecting selection on ancestry in admixed populations. Here we build off of the framework set up by Long 1991.
Contrasting ancestry at a locus to genome-wide expectations is the underlying strategy for other admixture-based selection scans that do not consider drift, e.g. genomic clines (CITE bayesian genomic cline literature), and approaches that account for drift in ancestry in single admixed populations, but do not compare across populations.

Our new comparative method for admixed populations extends a family of methods that scan the genome for loci showing an excess of divergence beyond the genome-wide expectation. Under the assumption that divergence patterns across most of the genome are shaped by drift alone, outliers beyond this expected neutral distribution are likely generated by divergent selection. Building from the theoretical framework set up by Cavalli-Sforza (CITE Cavalli-Sforza 1966 'Population structure and human evolution), Lewontin and Krakauer published the first of these methods ('LK' test), which tested for excess Fst between populations at candidate selected sites compared to the variance in Fst observed across all loci (CITE Lewontin and Krakauer 1973). Their approach assumes a simple star-like tree with equal branch lengths extending to each population, and creates false-positives when there are unequal population sizes, hierarchical structure, or uneven migration between populations (CITE Nei and Maruyama 1975 "Lewontin-Krakauer test for neutral genes; Lewontin and Krakauer 1975; Robertson 1975a and 1975b; MAYBE CITE MORE: Tsakas Krimbas 1976, Nei and Chakravarti 1977 and Nei 1977 are listed in Bonhomme for a similar point). More recently, population geneticists returned to this topic because whole-genome sequence data makes it possible to infer, and correct for, non-star like relationships between populations by calculating an empirical population kinship matrix 'F'. 
These more recent methods reduce false-positives by explicitly modeling unequal branches leading to populations that vary in N_{e} ('BAYESCAN' CITE Foll and Gaggiotti 2008), hierarchical tree-like population structure ('F-LK' CITE Bohnhomme 2010 10.1534/genetics.110.117275), and arbitrary population kinship relationships due to gene flow and drift ('xTx' CITE Coop 2010;  Guenther and Coop 2013).
A FEW SENTENCES ABOUT HAP-FLK ASSUMPTIONS VS A TRUE ADMIXTURE MODEL. WHY IT'S SO DIFFERENT.
In contrast, admixed populations may come from multiple admixture events and not share an ancestral admixed population. 

We additionally extend the environmental selection test proposed by Coop et al. 2010 ('Bayenv' CITE Coop 2010) to an admixture scenario. The original test identified linear relationships between allele frequencies and environmental variables, where the outcome and environment vector are both centered and transformed by the allelic kinship matrix. Here, we identify loci where ancestry is associated with particular environments by fitting a linear model between observed population ancestry frequencies and environmental variables, where the outcome and environment vectors are transformed by the ancestry kinship matrix.
 
One notable technical difference between Bayenv and ancestryEnv is that the independent variable in our linear model, ancestry at a focal locus, does not necessarily have mean zero across populations and therefore we are able to test for directional selection that raises or lowers the mean ancestry at a locus. In contrast, non-admixture methods such as Bayenv have no power to detect directional selection shared across the whole sample and can only capture divergent selection between populations. This limitation arises because the expected allele frequency for every population at any one locus under drift is simply the allele frequency at that locus in the shared ancestral population. Since we typically cannot observe this ancestral frequency, most inference methods estimate the ancestral allele frequency using the mean allele frequency across present-day populations. In other words, the allele frequencies are 'centered' around the present-day observed mean and we do not observed if this 'center' deviates from the true ancestral allele frequency. This approach does not detect strong allele frequency shifts affecting \textit{all} sampled populations, i.e. fixation of favorable alleles across populations or widespread parallel directional selection rather than divergent selection.

In contrast, for admixed populations, we have more information about the expected ancestry frequency at any given locus than just the observations at that locus in our sample. The expected ancestry frequency is the same across all loci in the genome and we can use a genome-wide estimate of the expected admixture proportion (sometimes called the 'hybrid index') to identify deviations at individual loci. This non-centered vs. centered difference means we can test for an excess or deficit of one ancestry across \textit{all} of our populations, which is equivalent to adding an intercept term to our linear model. We add this intercept term by estimating the slope for a vector of 1's, also transformed by the ancestry kinship matrix. We also include this kinship-transformed intercept when testing for environmental correlations, the intuition being that we may not have sampled symmetrically across the key environmental selection gradient. If we did not include this intercept, we would be assuming that the center of our environmental observations is the point where selection changes from negative to positive across the environmental gradient.

but instead the expected allele frequency at any locus under drift is the ancestral allele frequency at that locus and, in lieu of an ancestral sample, is estimated using the mean across present-day sampled populations. 


extra text:
Methods by Bonhomme and Coop refine the original LK test and reduce false-positives by incorporating the empirical interdependence between populations directly into their neutral models. Specifically, Bonhomme fit a hierarchically structured population tree matrix 'F' to the genomewide data ('F-LK' Bonhomme) and Coop et al. estimate a more general variance-covariance matrix, capturing arbitrary allelic relationships among populations (Coop 2010). Bonhomme et al. extended the original LK approach to account for additional expected variance due to hierarchical relationships among populations by incorporating the population kinship matrix F ('F-LK' test CITE Bonhomme 2010) and  


Bonhomme et al note a departure from theoretical chi-squared distribution due to extreme initial frequency values, and that their model fits well when 0.2 < p0 < 0.8 (CITE Bonhomme 2010 et al.).
Other methods have jointly estimated kinship and environmental effects in a mixed model which increases power to detect selection (Frichot et al 2013), possibly because the kinship matrix absorbs less of the selection signal (CITE observation from Guenther and Coop 2013). I could maybe do this too.

We extend this work by developing 
This has three main advantages:

In these previous population comparison methods, the 'center' or assumed ancestral allele frequency for any particular locus is the mean allele frequency across populations (although see ** for using an external source for ancestral allele frequencies). 

Bayenv, hapflk, pcadapt

Relationship to admixture models, e.g. bayesian clines.

Other hierarchical bayesian approaches jointly estimate the population-specific effects on allele frequency jointly across loci at the same time as estimating locus-specific effects of selection for each locus (CITE Balding 1996; Beaumont and Balding 2004; Foll and Gaggiotti 2008). Rather than a Bayesian framework, we calculate one covariance matrix and use that as a fixed population effect shared across all loci.

\subsection{Extension to test different models of selection}

The zTz statistic only captures a single distance from the neutral model, and as such, offers no information on which populations may be under selection, nor in which direction. Therefore, to test a particular selected locus' fit to different proposed models of selection, we apply methods from linear regression. 



\subsection{Relationship to phylogenetic contrast}
The main point of phylogenetic-contrast type methods, like ours, is to reduce Type I errors, or false positives, because correlation between data points that are assumed to be independent inflates the variance in model estimates beyond the assumed variance for confidence intervals and assessing model fit (e.g. slopes in a linear model). However, the slopes are not systematically reduced by applying the rotation or contrast. Both phylogenetic contrast methods and ordinary least squares generate unbiased estimates of the slope, or correlations between the data and an environmental data. For more information, see Rohlf 2006 "A Comment on Phylogenetic Correction" https://www.jstor.org/stable/pdf/4095344.pdf?refreqid=excelsior%3A37b064a12dbf07772d74b3c56e57f54c which sites Pagel 1993 for the 'unbiased estimators' assertion. So, the p-values are better calibrated, but the actual point estimates for the slopes will usually be more accurate in the contrast but could be more accurate for individual cases in the OLS framework (because both have some variance around the true value, so don't over-interpret differences between the two).
So the question really comes down to, how well calibrated are my p-values using zAnc-style rotations?
\end{document}
