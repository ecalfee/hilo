\documentclass[12pt]{report}
\usepackage{amsmath}
\begin{document}

\section{Methods}
These are the methods for the HiLo maize-mexicana admixture project.

\subsection{DNA extraction, library preparation, pooling/barcoding etc.}
\subsection{Sequencing}
Twelve** pairs of sympatric maize (landrace) and mexicana populations, twelve individuals per population, were genotyped using low-coverage whole-genome-sequencing (mean **x - **x). These populations span a range of altitudes from ***m to ***m in Mexico (see map, Fig. **). In addition, ten mexicana individuals from each of three allopatric populations were sequenced as non-admixed reference panel for mexicana ancestry and sequenced twice for additional depth (**x - **x mean individual coverage). Of the individuals sequenced in this study, ** were excluded due to low library amplification (< **x coverage), leaving a total of ** sympatric maize paired with ** sympatric mexicana and *** reference allopatric mexicana. For a maize reference panel, we use the publicly available 282 maize diversity panel which includes lines from *** and **. From these lines, we excluded ** and **. Mexicana is known from other sequencing efforts to be introgressed into maize from the highlands of Mexico and Guatemala, and to a lesser extent, the southwestern USA (Wang 2017). \par
\subsection{Alignment and Filtering}
We mapped reads using bwa to the AGPv4 maize genome ([bwa command; \& cite ABPv4]). For the reads that mapped (n= **, **\%), we applied additional filtering to remove reads with low mapping quality (n = **) [samtools 30?], remove likely PCR duplicates (n = **) [samtools not picard blah blah], and filter by adjusted base quality scores (BAQ > 20 in samtools)[NO local re-alignment around indels GATK (?)]. We then removed indels and sites that are invariant across our sequenced individuals and the maize reference panel. Of the total ** SNPs in the final variant set, ** vary within the maize reference panel only.
\subsection{Alignment and Filtering methods from Wang 2017}
Wang, Beissinger, Lorant, Ross-Ibarra, Ross-Ibarra, Hufford 2017. The interplay of demography and selection during maize domestication and expansion.\par
Sequenced 31 maize landraces from across the Americas and 4 parviglumis from Balsas River Valley, Mexico.\par
"Library preparation and Illumina HiSeq 2000 sequencing (100-bp paired-end) were conducted by BGI (Shenzhen, China) following their established protocols. the Burrows-Wheeler Aligner (BWA) v.0.7.5.a [73] was used to map reads to the maize B73 reference genome v3 (GenBank BioProject PRJNA72137) [74] with default settings. The duplicate molecules in the realigned bam files were removed with MarkDuplicates in Picardtools v.1.106 (http://broadinstitute.github.io/picard), and indels were realigned with the Genome Analysis Toolkit (GATK) v.3.3-0 [75]. Sites with mapping quality less than 30 and base quality less than 20 were removed, and only uniquely mapped reads were included in downstream analyses."\par
They have high 24-53X coverage per individual, which may make some SNP calling methods better for their study that don't apply to the current study. We will not be able to phase haplotypes and I am not sure that HaplotypeCaller is an option with low coverage data because it's like a local assembly (also newest release says you don't need to do GATK indel re-alignment if you do HaplotypeCaller method).\par
"SNPs were called via HaplotypeCaller and filtered via VariantFiltration in GATK [75] across all samples. SNPs with the following metrics were excluded from the analysis: QD <2.0; FS >60.0; MQ <40.0; MQRankSum <−12.5; ReadPosRankSum <−8.0. Vcftools v.0.1.12 [76] was used to further filter SNPs to include only bi-allelic sites. Following these data filtering steps, our data set consisted of 49 million SNPs. SNPs were phased using BEAGLE v.4.0 [77] with SNP data from the maize HapMap2 panel [78] used as a reference. Only sites with depth between half and twice of the mean depth were included in analyses. In addition, the software SNPable (http://lh3lh3.users.sourceforge.net/snpable.shtml) was used to mask genomic regions in which reads were not uniquely mapped. The mappability mask file for MSMC was generated by stepping in 1-bp increments across the maize genome to generate 100-bp single-end reads, which were then mapped back to the maize B73 reference genome [74]. Sites with the majority of overlapping 100-mers mapped uniquely without mismatch were determined to be “SNPable” sites and used for the MSMC analyses."\par

\subsection{Ancestry inference}
To infer global ancestry genomewide and local ancestry along the genome, we used NGSadmix and ancestry\_hmm respectively, chosen because they account for the genotype uncertainty from low-coverage sequencing. We used a subset of the total SNPs to infer ancestry because these methods assume independent information about ancestry from each marker and therefore markers in linkage disequilibrium within ancestry confound inference.\par
For global ancestry inference, randomly subsample SNPs to a set of 25k with high coverage across individuals and maf > .01 and input beagle genotype likelihoods for these spots from all individuals sequenced for this study to NGSadmix. We can repeat this procedure with 10 different SNP subsets to confirm similar results. This process doesn't include the maize reference or indicate which individuals are the unadmixed teosinte and I can use it to get a second estimate of allele frequencies in maize and do PCR with that against the reference population. We expect a gradient of admixture from low mexicana ancestry in the lowlands and high mexicana ancestry in the highlands and little to no maize ancestry in the mexicana allopatric panel.\par

For local ancestry inference, we need more SNPs, but still with low LD in the ancestral population. However, rather than a random subset, we want to keep SNPs with high ability to distinguish between mexicana and maize ancestry. To reduce the total number of SNPs, we first filtered out SNPs with low ancestry informativeness (difference in frequency between mexicana and maize < 0.2). From the remaining ***number** markers, we reduced our SNPs by using a greedy algorithm in plink that removes SNPs in high LD in a sliding window of 50 variants ([plink code snipet here]), resulting in a final variant set of *n=** SNPs. Then we input allele counts for reference populations and read counts supporting alternate and reference alleles for each admixed individual and run ancestry\_hmm for one admixed population at a time with whole-genome data. We estimate time since admixture to be at least 1 generations ago and at most 9000 generations in the past at the time of maize domestication assuming 1 gen/year (Matsuoka 2002) [**alt I could use 200 generations as the upper limit because 170ish is the upper of what Hufford 2013 found**], using the mean global ancestry proportion from NGSadmix as the proportion of migrants entering the population. Recombination distance between markers was calculated using the 0.2cM scale recombination map produced from 25 families in the maize Nested Association Mapping (NAM) population (Ogut 2015).\par

Our allopatric mexicana population was not sequenced to a high enough depth to accurately call heterogygous vs. homozygous genotypes. Therefore, to estimate an unbiased allele frequency, we randomly sampled one read at each SNP from each low-coverage allopatric mexicana individual and used these as our reference allele counts, essentially creating **number** unbiased pseudo-haplotypes. The maize reference allele counts are more certain and taken from the 282 panel with weighting by phylogenetic dissimilarity (?) to reduce the impact of biased sampling.\par

We assess confidence in our ancestry calls by calculating and comparing divergence to our reference panels in regions with high confidence mexicana or maize ancestry. This is a partially independent assessment from the local ancestry calling because we use all SNPs, **\% more than were used in ancestry\_hmm due to filtering for high LD [see supplemental fig. X for posterior prob. maize <-> mex on x axis and mean divergence on y axis, with separate lines for maize and mexicana .. I could also do this for divergence to specific mexicana groups] \par

\end{document}